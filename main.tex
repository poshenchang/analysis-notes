% Version 7.4 (22-08-18)
\documentclass[11pt, a4paper]{report}
% \usepackage[top=1.2in, bottom=1.2in, left=1.23in, right=1.23in, showframe]{geometry}
\usepackage[top=1.2in, bottom=1.2in, left=1.03in, right=1.03in]{geometry}

%%%%%%%% Additional (Required) Usepackages {
\usepackage[x11names]{xcolor}
\usepackage{adjustbox}
\usepackage{algorithm}
\usepackage{amsmath,amsthm,amssymb}
\usepackage[toc,page]{appendix}
\usepackage{array, makecell}
\usepackage{authblk}
% \usepackage[english]{babel} %dangerous
\usepackage{bbm}
\usepackage{blindtext}
\usepackage{bm}
\usepackage{cancel}
\usepackage{censor}
\usepackage{collcell}
\usepackage{comment}
\usepackage[shortlabels]{enumitem} 
\usepackage{epigraph}
\usepackage{esvect}
\usepackage{etoolbox}
\usepackage{fancyhdr}
\usepackage{fontspec}
\usepackage[linguistics]{forest}
\usepackage[bottom]{footmisc}
\usepackage{graphicx}
\usepackage[unicode=true, pdfborder={0 0 0},  bookmarksdepth=-1]{hyperref}  
\usepackage{indentfirst}
\usepackage{index}
\usepackage{lipsum}
\usepackage{listings}
\usepackage{mathtools}
\usepackage{multicol}
\usepackage[square, comma, numbers, super, sort&compress]{natbib}
\usepackage{nicematrix}
\usepackage{pdfpages}
\usepackage{pifont}
\usepackage{polyglossia}
\usepackage{ragged2e}
\usepackage{scalerel}
\usepackage{setspace}
\usepackage{seqsplit}
\usepackage{tabularx}
\usepackage{tcolorbox}
\usepackage{thmtools}
\usepackage{tikz} % graph
\usetikzlibrary{automata,arrows,positioning}
\usepackage{titlesec}
\usepackage{titletoc}
\usepackage{titling}
\usepackage{type1cm}	 
\usepackage[normalem]{ulem}
\usepackage{wrapfig}
\usepackage{xeCJK}
\usepackage{xparse}
\usepackage{xstring}
\usepackage{yhmath}
\usepackage{zhnumber}


\usepackage{chessboard}
\usepackage{xskak}
\usepackage[noend]{algpseudocode}

% packages that need to be loaded after other packages
\usepackage[style=iso]{datetime2}

% tikz settings

\tikzset{
  every state/.style = {thick},
}

%%%%%%%% Length/Name Settings {

\setlength{\headheight}{15pt}  
\setlength{\droptitle}{-1cm}
\setlength{\headsep}{0.3cm}
\setlength{\bibsep}{0pt minus 0.4ex}
\setlength{\parskip}{0.8em}
\renewcommand{\baselinestretch}{1.2}

\setlength{\parindent}{2em}

\renewcommand{\contentsname}{目錄} 
\renewcommand{\refname}{References}
\renewcommand{\abstractname}{ABSTRACT}

%}
%%%%%%%% Theorems {

\tcbuselibrary{theorems}
\tcbuselibrary{breakable}
\tcbset{
    exstyle/.style={
        standard jigsaw, opacityback=0, colframe=black, boxrule=0.75pt, sharp corners=all, 
        fonttitle=\rmfamily\mdseries\color{black}, titlerule=0pt, opacitybacktitle=0, toptitle=2mm, 
        separator sign dash, description delimiters={}{},
        middle=2mm,
        breakable
    },
    thmstyle/.style={
        standard jigsaw, opacityback=0, colframe=black, boxrule=0.75pt, sharp corners=all,
        fonttitle=\rmfamily\mdseries\color{black}, attach title to upper, after title={\enspace}, 
        separator sign={\thickspace}, description delimiters={(}{)},
        lower separated=false, halign lower=justify,
        breakable
    },
}

\newtcbtheorem[number within=section]{proa}{Problem}%
    {
    standard jigsaw, opacityback=0, colframe=black, fonttitle=\bfseries\color{black}, boxrule=0.75pt, arc=0mm, separator sign dash, description delimiters={}{}, detach title, after upper={\par\hfill\tcbtitle}}{proa}

% \newtcbtheorem[number within=section]{prob}{\bfseries Problem}%
%   {
%     standard jigsaw, opacityback=0, colframe=black, boxrule=0.75pt, sharp corners=all, 
%     fonttitle=\rmfamily\mdseries\color{black}, titlerule=0pt, opacitybacktitle=0, toptitle=2mm, separator sign dash, description delimiters={}{},
%     middle=2mm, lower separated=true, halign lower=flush right
% }{prob}

% \newtcbtheorem[number within=section]{proc}{\bfseries Problem}%
%   {
%     standard jigsaw, opacityback=0, colframe=black, boxrule=0.75pt, sharp corners=all,
%     fonttitle=\rmfamily\mdseries\color{black}, attach title to upper, after title={\enspace}, 
%     separator sign={\thickspace}, description delimiters={(}{)},
%     middle=5pt, lower separated=false, halign lower=flush right
% }{proc}

\newtcbtheorem[number within=section]{example}{\bfseries Example}{thmstyle}{exm}
\newtcbtheorem[use counter from=example]{exercise}{\bfseries Exercise}{thmstyle}{exc}

\newtcbtheorem[number within=section]{prob}{\bfseries Example}{exstyle}{prob}

\newtcbtheorem[number within=section]{proc}{\bfseries Problem}{thmstyle}{proc}


\newtcbtheorem[number within=section]{theorem}{\bfseries Theorem}{exstyle}{theorem}
\newtcbtheorem[use counter from=theorem]{definition}{\bfseries Definition}{exstyle}{definition}
\newtcbtheorem[use counter from=theorem]{property}{\bfseries Property}{exstyle}{property}
\newtcbtheorem[use counter from=theorem]{lemma}{\bfseries Lemma}{exstyle}{lemma}
\newtcbtheorem[use counter from=theorem]{corollary}{\bfseries Corollary}{exstyle}{corollary}


\newcommand{\resetcounters}
    {
    \setcounter{section}{0}
    \setcounter{ans}{0}
    \setcounter{ax}{0}
    \setcounter{con}{0}
    \setcounter{ex}{0}
    \setcounter{pr}{0}
    \setcounter{pp}{0}
    \setcounter{prop}{0}
    \setcounter{cj}{0}
}

% 如果你/妳自定義新的 theorem, 記得回來更新 \resetcounters 的定義

\newtheoremstyle{one}   % name of the style to be used
  {8.5pt}                 % measure of space to leave above the theorem. E.g.: 3pt
  {12.5pt}                % measure of space to leave below the theorem. E.g.: 3pt
  {}                    % name of font to use in the body of the theorem
  {}                    % measure of space to indent
  {\bfseries}           % name of head font
  {}                    % punctuation between head and body
  {1em}                 % space after theorem head; " " = normal interword space
  {}                    % ???
\theoremstyle{one}

\newtheorem{ans}{Answer}[section]
\newtheorem{ax}{Axiom}[section]
\newtheorem{cj}{Conjecture}[section]
\newtheorem{con}{Conclusion}[section]
\newtheorem*{clm}{Claim}
\newtheorem{pr}{Problem}[section]
\newtheorem{prop}{Proposition}[section]
\newtheorem*{rmk}{Remark}

\newtheorem{ex}{Example}[section]
\newtheorem{exs}[ex]{Exercise}

\newtheorem{pp}{Property}[section]
\newtheorem{thm}[pp]{Theorem}
\newtheorem{df}[pp]{Definition}
\newtheorem{lm}[pp]{Lemma}
\newtheorem{cl}[pp]{Corollary}

% --- Proof, Solution, Sketch, Hint, Remark (ZH)

\newenvironment{pf}[1][0]
 {%
  \begin{proof}[\textit Proof\ifnum#1>0 \, #1\fi.]
 }{\end{proof}}

\newenvironment{sol}[1][0]
 {%
  \renewcommand{\qedsymbol}{}
  \begin{proof}[\textit Solution\ifnum#1>0 \, #1\fi.]
 }{\end{proof}}

\newenvironment{sketch}[1][0]
 {
  \renewcommand{\qedsymbol}{}
  \begin{proof}[\textit Sketch\ifnum#1>0 \, #1\fi.]
 }{\end{proof}}

\newenvironment{hint}{\noindent\mybf{提示.}\hspace{1ex}}{\hfill}




% \newtcolorbox{mybox}[2][]{colbacktitle=red!10!white, colback=blue!10!white, coltitle=red!70!black, title={#2}, fonttitle=\bfseries,#1}


% --- Theorems in the enumerate environment

%}
%%%%%%%% New Commands {

\newcommand{\mybf}[1]{{\bfseries #1}}
\renewcommand{\textbf}{\mybf}


% --- Geometry
\newcommand{\Arc}[1]{\wideparen{{#1}}}
\newcommand{\degree}{^\circ}
\newcommand{\Line}[1]{\overleftrightarrow{{#1}}}
\newcommand{\np}[1]{\\[{#1}] \indent}
\newcommand{\Ray}[1]{\overrightarrow{{#1}}}
\newcommand{\Segment}[1]{\overline{{#1}}}

% --- Operatornames
\newcommand{\adj}{\operatorname{adj}}
\newcommand{\csch}{\operatorname{csch}}
% \newcommand{\d}{\mathrm{d}}
\newcommand{\diag}{\operatorname{diag}}
\newcommand{\diam}{\operatorname{diam}}
\newcommand{\diff}{\operatorname{diff}}
\newcommand{\dis}{\operatorname{dis}}
% \newcommand{\ddx}[1]{\frac{\mathrm d^{#1}}{\mathrm d x^{#1}}}{}
\newcommand{\lcm}{\operatorname{lcm}}
\newcommand{\li}{\operatorname{li}}
\newcommand{\ord}{\operatorname{ord}}
\newcommand{\pow}[2]{\mathbf{Pow}_{#1}(#2)}
\newcommand{\per}{\operatorname{per}}
\newcommand{\Per}{\operatorname{Per}}
\newcommand{\rank}{\operatorname{rank}}
\newcommand{\sech}{\operatorname{sech}}
\newcommand{\sgn}{\operatorname{sgn}}
\newcommand{\shl}{\mathsf{sh}_L} % shift left function
\newcommand{\shr}{\mathsf{sh}_R} % shift right function
\newcommand{\sh}{\mathsf{sh}}
\newcommand{\Span}{\operatorname{span}}
\newcommand{\transpose}[1]{{#1}^\text{T}}

\newcommand{\R}{\mathbb R}
\newcommand{\p}{\partial}
\renewcommand{\d}{\mathrm{d}}
\newcommand*\circled[1]{\tikz[baseline=(char.base)]{
            \node[shape=circle,draw,inner sep=2pt] (char) {#1};}}
\NewDocumentCommand{\ddx}{o}{
    \IfNoValueTF{#1}{\frac{\mathrm d}{\mathrm d x}}{\frac{\mathrm d^{#1}}{\mathrm d x^{#1}}}
}


% % --- emojis

% \newcommand{\mystar}{\,\raisebox{-1.15pt}{\makebox[1.8ex][c]{\includegraphics[width=2.2ex]{Five-pointed_star_copy.svg.png}}}\,}
% \newcommand{\mystarempty}{\,\raisebox{-1.15pt}{\makebox[1.8ex][c]{\includegraphics[width=2.2ex]{Five-pointed_star.svg.png}}}\,}

% --- Poison Geometry
\newcommand{\pole}[2]{\mathbb{P}_{\mathcal{#1}} (#2)}
\newcommand{\tg}[2]{\mathcal{T}_{\mathcal{#1}} (#2)}

\let\oldemptyset\emptyset
\let\emptyset\varnothing

%For fills the dot inside \odot gradually
\setlength{\unitlength}{1em}
\newcommand\like[1]{\begin{picture}(1,1)
\ifnum0=#1\put(.5,.35){\circle{1}}\else
\ifnum10=#1\put(.5,.35){\circle*{1}}\else
\put(.5,.35){\circle{1}}\put(.5,.35){\circle*{.#1}}
\fi\fi\end{picture}}
% e.g. \like{1}

%optional argument for matrix environments
\makeatletter
\renewcommand*\env@matrix[1][*\c@MaxMatrixCols c]{%
  \hskip -\arraycolsep
  \let\@ifnextchar\new@ifnextchar
  \array{#1}}
\makeatother

% nice matrix settings \usepackage{nicematrix}
\NiceMatrixOptions{
    code-for-first-row = \color{blue} ,
    code-for-last-row = \color{blue} ,
    code-for-first-col = \color{blue} ,
    code-for-last-col = \color{blue},
    cell-space-limits = 3pt
}

% epigraphs

% \epigraphsize{\small}% Default
\setlength\epigraphwidth{12cm}
\setlength\epigraphrule{0pt}

\makeatletter
\patchcmd{\epigraph}{\@epitext{#1}}{\fontspec{CMU Sans Serif Demi Condensed}\@epitext{#1}}{}{}
\makeatother

%}
%%%%%%%% Big Environments {

%changemargin
\def\changemargin#1#2{\list{}{\rightmargin#2\leftmargin#1}\item[]}
\let\endchangemargin=\endlist
% e.g. \begin{changemargin}{1in}{1in}

% wrapfig
\newsavebox\curwrapfig
\makeatletter
\long\def\wrapfiguresafe#1#2#3{%
  \sbox\curwrapfig{#3}%
  \par\penalty-100%
  \begingroup % preserve \dimen@
    \dimen@\pagegoal \advance\dimen@-\pagetotal % space left
    \advance\dimen@-\baselineskip % allow an extra line
    \ifdim \ht\curwrapfig>\dimen@ % not enough space left
      \break%
    \fi%
  \endgroup%
  \begin{wrapfigure}{#1}{#2}%
    \usebox\curwrapfig%
  \end{wrapfigure}%
}
\makeatother




%}
%%%%%%%% Langs {

\renewcommand\theadfont{\bfseries}

% --- The Latin Family ---

% \setdefaultlanguage{english}
% \setotherlanguage[numerals=hebrew]{hebrew}

% \setmainfont{CMU Serif}
% \setsansfont{TeX Gyre Heros}
\setmonofont{SourceCodePro}[
    Path=./SourceCodeProFiles/,
    Scale=1,
    Extension = .ttf,
    UprightFont=*-Regular,
    BoldFont=*-Bold,
    ItalicFont=*-Italic,
    BoldItalicFont=*-BoldItalic
    ]


\newfontfamily\robotolight[Path = ./fonts/]{Roboto-Light}
\newfontfamily\robotobold[Path = ./fonts/]{Roboto-Bold}
\newfontfamily\garet[Path = ./fonts/]{Garet-Book}
% \newfontfamily\arial{Arial}
% \newfontfamily\myfont{Times New Roman}

% Recommand for Greek
% \newfontfamily\garamound{Garamond Libre}
% \newfontfamily\lato{Lato}
% \newfontfamily\notoserif{Noto Serif Display}

% --- Chinese (Traditional/Simplified) ---

% \setCJKmainfont{TW-Kai}
% \setCJKsansfont{Noto Sans CJK TC}
% \setCJKmonofont{Noto Serif CJK TC}

\defaultCJKfontfeatures{AutoFakeBold=3,AutoFakeSlant=.4}

% \newCJKfontfamily\twkai{TW-Kai}
% \newCJKfontfamily\cjkserif{Noto Serif CJK TC}
% \newCJKfontfamily\cjksans{Noto Sans CJK TC}

% \newCJKfontfamily\song{cwTeXFangSong}
% \newCJKfontfamily\cwtexming{cwTeXMing}

% --- Japanese ---

% \newCJKfontfamily\japsans{IPAGothic}
% \newCJKfontfamily\japprint{Oradano-mincho-GSRR}

\XeTeXlinebreaklocale "zh"
\XeTeXlinebreakskip = 0pt plus 1pt

%}

\titleformat{\chapter}[display]{}{\garet\LARGE\bfseries CHAPTER \thechapter}{0.5em}{\Huge\bfseries}

\titleformat{\section}[hang]{\normalfont\Large\bfseries}{\thesection}{1em}{\Large\bfseries}

\titleformat{\subsection}{\normalfont\large\bfseries}{}{1.20em}{\large }[\vskip-21pt{\makebox[\linewidth][l]{\rule{0.6\textwidth}{0.75pt}}}]

% single character
\newcommand{\N}{{\mathbb N}}
\newcommand{\Q}{{\mathbb Q}}
\newcommand{\Z}{{\mathbb Z
}}
\renewcommand{\R}{{\mathbb R}}
\newcommand{\C}{{\mathbb C}}
\newcommand{\F}{{\mathbb F}}
\renewcommand{\L}[2]{{\mathcal L}\left(#1, #2\right)}
\newcommand{\M}[3]{{\mathsf M}_{#1\times #2}\left({#3}\right)}

\renewcommand{\Hat}[1]{\widehat{#1}}

% abbreviation
\newcommand{\ctr}{\rightarrow\!\leftarrow} % contradiction
\newcommand{\To}[1][\;]{\xrightarrow{\;\; #1 \;\;}}
\newcommand{\Mapsto}[1][\;]{\xmapsto{\;\; #1 \;\;}}
\newcommand{\op}[1]{\operatorname{#1}}
\newcommand{\ags}[1]{ %span
    \left\langle #1\right\rangle
}
\newcommand{\norm}[1]{\lVert #1 \rVert}
\newcommand{\rng}[4][1]{{#3}_{#1} #4 \cdots #4{#3}_{#2}}
\newcommand{\set}[1]{{\left\{{#1}\right\}}}
\newcommand{\abs}[1]{{\left| #1 \right|}}

\NewDocumentCommand{\OP}{u{ }}{\operatorname{#1}}
\let~\OP

% pictures
\begin{comment}
    \newcommand{}{\text{\raisebox{-0.5ex}{
        \includegraphics[width = 0.5cm]{Icons/}
        }}
    }
\end{comment}
\newcommand{\flower}[1][0.2]{{\includegraphics[width=#1cm]{Icons/flower.png}}}
\newcommand{\car}[1][0.2]{{\includegraphics[width=#1cm]{Icons/car.png}}}
\newcommand{\eye}{\includegraphics[width=0.5cm]{Icons/eye.jpeg}}
\newcommand{\flag}{\includegraphics[width=0.5cm]{Icons/flag.jpeg}}
\newcommand{\mystar}{\,\raisebox{-1.15pt}{\makebox[1.8ex][c]{\includegraphics[width=2.2ex]{Five-pointed_star_copy.svg.png}}}\,}
\newcommand{\mystarempty}{\,\raisebox{-1.15pt}{\makebox[1.8ex][c]{\includegraphics[width=2.2ex]{Five-pointed_star.svg.png}}}\,}
\newcommand{\coni}{\includegraphics[width=0.5cm]{Icons/Conical_flask.jpeg}}

% format
\newenvironment{stm}[1]{ % statement
    \renewcommand{\qedsymbol}{}
    \begin{proof}[\emph{\underline{#1}}]
}{\end{proof}}

\newenvironment{emo}[1]{ % emoji
    \renewcommand{\qedsymbol}{}
    \begin{proof}[\emph{{#1}:}\nopunct]
 }{\end{proof}}

\renewcommand{\Tilde}{\widetilde}
% useful built-in
\begin{comment} % multiline inside display math mode

\begin{split}
     & \\
     &
\end{split}

\end{comment}

\begin{comment} % multiline inside math mode (including inline)

\begin{aligned}
    & \\
    &
\end{aligned}

\end{comment}

\begin{comment} % placing text between equations

\begin{align}
    \\
    \intertext{}
    \\
\end{align}

\end{comment}

\begin{comment} % matrix with grid lines

\left(\begin{array}{l|c:r}
     & & \\
     \hline
     & & \\
     \hdashline
     & &
\end{array}\right)

\end{comment}

\begin{comment} % circled number

\begin{enumerate}[label = \protect\circled{\arabic*}]
    \item 
\end{enumerate}

\end{comment}

% for analysis
\newenvironment{circenum}[1][\arabic*]
 {\begin{enumerate}[label = \protect\circled{#1}]
 }{\end{enumerate}}
\NewDocumentEnvironment{circenum*}{O{\qquad}O{\arabic*}}
 {\begin{center}\begin{enumerate*}[label = \protect\circled{#2}, itemjoin = #1]
 }{\end{enumerate*}\end{center}}
\renewcommand{\c}{\complement}
\renewcommand{\u}{\mathcal{U}}
\newcommand{\X}{\fancyRoman{10}}
\newcommand{\Int}{\op{Int}}

\renewcommand{\Bar}[1]{\overline{#1}}
\newcommand{\oneone}{\hookrightarrow}
\newcommand{\onto}{\twoheadrightarrow}
\newcommand{\bij}{\mathrel{\hookrightarrow\hspace{-1.8ex}\to}}
\newcommand{\isom}{\xrightarrow{\;\;\sim\;\;}}
\NewDocumentCommand{\maps}{mO{\To}mO{\;}mm}{
    \begin{aligned}
        #1 #2[#4] & \; #3\\
        #5 \Mapsto[\phantom{#4}] & \; #6
    \end{aligned}
}
\usepackage{centernot}
\newcommand{\notimplies}{\centernot\implies}

\color{white}
\pagecolor{black}
\newcommand{\eps}{\varepsilon}

\begin{document}
\pagestyle{fancy} % e.g. plain, empty, fancy
\lhead{}
\chead{} 
\rhead{} 
\lfoot{} 
\cfoot{} 
\rfoot{\thepage}

\renewcommand{\headrulewidth}{0.4pt}
\renewcommand{\footrulewidth}{0.4pt}

\setlist[enumerate]{topsep=0pt}
\setlist[itemize]{topsep=0pt}

%%%%%%%%%%%%%%%%%%%%%%%%%%%%%%%%%%%%%%%%%%%



\begin{titlepage}
\flushright
\vspace*{7cm}

{\garet\fontsize{48}{54}\bfseries
\text{Analysis I} %main title
}

\vspace{0.2cm}
% \Large{\cjkserif{}} %subtitle
   
\vspace{2 cm}
\Large{by Poshen Chang}%your name

\large{v.1}%YOUR NUMBER

\vspace{3.5 cm}
\Large{\today}%date

% \vspace{0.25 cm}
% \Large{\myfont }%place

\vfill
\end{titlepage}

\tableofcontents

\resetcounters
\justifying
% \thispagestyle{fancy}
\setlength{\parindent}{2em}

% \newtcolorbox{mybox}[2][]{}

% \tcbset{after title={\hfill\colorbox{black}{\mdseries\color{white} solved}}}

% 以下輸入正文
%\setcounter{chapter}{0}
\tableofcontents

\chapter{Topological Spaces}

In the language of calculus, we defined convergence, continuity and such concepts on the real numbers $\R$. We would like to generalize this concept.

\section{Metric Spaces}

\begin{df}
    A metric space is a nonempty set $M$, together with a metric $d : M\times M\to \R$. $d(x, y)$ is a real number defined for all $x, y\in M$, which can be thought of as the distance between $x$ and $y$. $d$ satisfies the following properties:
    \begin{enumerate}[label={\alph*)}]
        \item (Positive definiteness) $d(x, y) \geq 0$, and $d(x, y) = 0$ iff $x = y$.
        \item (Symmetry) $d(x, y) = d(y, x)$.
        \item (Triangle inequality) $d(x, z) \leq d(x, y) + d(y, z)$.
    \end{enumerate}
    We say that the pair $(M, d)$ is a metric space. The metric $d$ can be omitted if it is clear from the context.
\end{df}

\begin{ex}
    The following are some examples of metric spaces:
    \begin{itemize}
        \item $(\R, d)$, where $d(x, y) = |x - y|$.
        \item $(\R^n, d)$, where
        \[
        d(\mathbf x, \mathbf y) = \sqrt{(x_1 - y_1)^2 + \cdots + (x_n + y_n)^2}.
        \]
        \item $(M, d)$, where $M\neq\emptyset$, and
        \[
        d(x, y) = \begin{cases}
            0, \quad \text{if } x = y, \\
            1, \quad \text{if } x \neq y.
        \end{cases}
        \]
        This metric is called the \textbf{discrete metric} on $M$.
    \end{itemize}
\end{ex}

If $\emptyset \neq A \subseteq M$, and if $M$ is a metric, then $(A, d)$ is also a metric space. We call $A$ a metric subspace of $M$, or $A$ inherits its metric from $M$.

\begin{df}
    We say that a sequence $(x_n)$ in $M$ converges to the limit $x$ in $M$ if for any given $\epsilon > 0$, $\exists N \in \N$ such that $n\geq N$ and $n \in \N$ implies $d(x_n, x) < \epsilon$.
\end{df}

It's easy to check that limits are unique. Also, every subsequence of a convergent sequence converges, and it converges to the same limit as the original sequence.

\section{Continuity}

\begin{df}
    Let $(M, d_M)$ and $(N, d_N)$ be metric spaces. We say that $f: M\to N$ is continuous if it preserves sequential convergence, i.e. for each $(x_n)$ in $M$ which converges to $x$ in $M$, the image sequence $(f(x_n))$ converges to $f(x)$.
\end{df}

\begin{prop}
    Composition of continuous functions is continuous
    \begin{proof}
        Let $M, N, P$ be metric spaces, and let $f: M\to N$, $g: N\to P$ be continuous functions.
        
        Let $(x_n)$ be a convergent sequence in $M$ with limit $x$. We have
        \[
        \lim_{n\to\infty} x_n = x \xRightarrow{f\text{ conti.}} \lim_{n\to\infty} f(x_n) = f(x) \xRightarrow{g\text{ conti.}} \lim_{n\to\infty} g(f(x_n)) = g(f(x)),
        \]
        thus $g\circ f$ is continuous.
    \end{proof}
\end{prop}

\begin{ex}
    The following are some examples of continuous functions:
    \begin{itemize}
        \item The identity map $~id : M\to M$ is continuous.
        \item Every constant function $f: M\to N$ is continuous.
        \item Every function $f: M\to N$ is continuous if $M$ is equipped with the discrete metric.
    \end{itemize}
\end{ex}

\begin{df}
    If $f: M\to N$ is a bijection such that $f$ and $f^{-1}: N\to M$ are continuous, then we say that $f$ is a homeomorphism. If there exists a homeomorphism between $M$ and $N$, we say that $M, N$ are homeomorphic, denoted by $M\cong N$.
\end{df}

Intuitively, a homeomorphism is a bijection that can bend, twist, stretch the space $M$ to make it coincide with $N$, but it cannot rip, puncture or shred $M$ etc.

\begin{ex}
    Let $\mathbb S^1$ be the unit circle in the plane. Consider the interval $[0, 2\pi)$. Define $f: [0, 2\pi) \to \mathbb S^1$ to be the function $f(\theta) = (\cos\theta, \sin\theta)$. $f$ is continuous and bijective, but $f^{-1}$ is not continuous (consider a sequence in $\mathbb S^1$ approaching $(1, 0)$ from the lower plane).
\end{ex}

\begin{prop}
    $f: M\to N$ is continuous iff it satisfies the following: $\forall \epsilon$ and $x\in M$, $\exists \delta > 0$ such that if $y\in M$ and $d_M(x, y) < \delta$ then $d_N(f(x), f(y)) < \epsilon$.
    \begin{proof}
        "$\implies$": Suppose that $f$ fails to satisfy the $\epsilon$-$\delta$ condition at some $x\in M$. $\exists \epsilon > 0$ such that $\forall \delta > 0$, $\exists y\in M$ such that $d_M(x, y) < \delta$ but $d_N(f(x), f(y)) \geq \epsilon$. Take $\delta = \frac 1n$. By our assumption, we can obtain a sequence $(y_n)$ with $d_M(x, y_n) < \frac 1n$ but $d_N(f(x), f(y_n)) \geq \epsilon$, so $(y_n)$ converges to $x$ but $f(y_n)$ does not approach $f(x)$, contradicting the continuity of $f$.

        "$\impliedby$": Suppose that $f$ satisfies the $\epsilon$-$\delta$ condition at $x$. Let $(x_n)$ be a sequence in $M$ susch that $(x_n)$ converges to $x$. Let $\epsilon > 0$. $\exists \delta > 0$ such that $d_M(x, y) < \delta \implies d_N(f(x), f(y)) < \epsilon$. Since $(x_n)$ approaches $x$, $\exists K \in \N$ such that if $n\geq K$, then $d_M(x_n, x) < \delta$, hence $d_N(f(x_n), f(x)) < \epsilon$ for all $n\geq K$. That is, $(f(x_n))$ converges to $f(x)$.
    \end{proof}
\end{prop}

\section{The Topology of a Metric Space}

\begin{df}
    Let $M$ be a metric space, $S\subseteq M$.
    \begin{itemize}
        \item We say $S$ is open if for each $x\in S$, $\exists r > 0$ such that $d(x, y) < r$ implies $y\in S$.
        \item We say $S$ is closed if its complement is open.
        \item We say that a point $x\in M$ is a limit of $S$ if there exists a sequence in $S$ that converges to $x$.
    \end{itemize}
\end{df}

\begin{prop}
    A set $S\subseteq M$ is closed iff it contains all its limits.
    \begin{proof}
        "$\implies$": Suppose that $S$ is closed. Let $(x_n)$ be a sequence in $S$ such that $(x_n)$ converges to $x$, $x\in M$. Suppose $x\notin S$, then since $S^c$ is open, $\exists r>0$ such that $d(x, y) < r$ implies $y\in S^c$. Since $(x_n)$ converges to $x$, we have $d(x_n, x) < r$ for all $n$ large enough, implying that $x_n \in S^c$, contradiction. Therefore, for any limit point $x$ of $S$, $x\in S$.

        "$\impliedby$": Suppose that $S$ contains all its limits. If $S^c$ is not open, then $\exists x \in S^c$ such that $\forall n\in \N$, $\exists x_n\in (S^c)^c = S$ such that $d(x, x_n) < \frac1n$. We have now constructed q sequence $(x_n)$ in $S$, but converges to a point in $S^c$, contradiction. Thus $S^c$ is open, i.e. $S$ is closed.
    \end{proof}
\end{prop}

\begin{rmk}
    Sets can be neither open nor closed, or they can also be both open and closed.
\end{rmk}

\begin{df}
    The topology $\mathcal T$ of $M$ is the collection of all open subsets of $M$.
\end{df}

\begin{prop}
    $\mathcal T$ is closed under arbitrary union, finite intersection, and $\mathcal T$ contains $\emptyset$ and $M$.

    \begin{proof}
        Clearly, $\emptyset$ and $M$ are open.

        Let $(\mathcal U_\alpha)$ be a collection of open subsets of $M$. Define $V = \bigcup_\alpha \mathcal U_\alpha$. For any $x\in V$, $x\in \mathcal U_\alpha$ for some $\alpha$. Since $\mathcal U_\alpha$ is open, $\exists r>0$ such that $d(x, y)<r \implies y\in \mathcal U_\alpha \subseteq V$, so $V$ is open.

        Define $W = \bigcap_{i=1}^n \mathcal U_i$. Given $x\in W$, for each $1\leq i\leq n$, $\exists r_i > 0$ such that $d(x, y) < r_i \implies y\in \mathcal U_i$. Take $r = \min{r_1, \ldots, r_n}$, then for any $y$ satisfying $d(x, y) < r$, we have $y\in \mathcal U_i$ for all $1\leq i\leq n$, thus $y\in W$. Hence $W$ is open.
    \end{proof}
\end{prop}

\begin{df}
    Let $X$ be a set. A topology $\mathcal T$ of $X$ is a collection of subsets of $X$ that satisfies the following:
    \begin{enumerate}[label={\alph*)}]
        \item $\mathcal T$ is closed under arbitrary union,
        \item $\mathcal T$ is closed under finite intersection, and
        \item $\emptyset, X \in \mathcal T$.
    \end{enumerate}
    We say that  $(X, \mathcal T)$ is a topological space if $\mathcal T$ is a topology of $X$. The elements of $\mathcal T$ are called open sets. We define $S\subseteq X$ to be closed if $S^c$ is open.
\end{df}

\begin{ex}
    The following are some examples of topological spaces:
    \begin{itemize}
        \item A metric space is a topological space.
        \item Let $X$ be a set, and $\mathcal T = \{\emptyset, X\}$. Then $(X, \mathcal T)$ is a topological space, which is know as the trivial topology.
        \item Let $X$ be a set, and let $\mathcal T$ be the power set of $X$, then $(X, \mathcal T)$ is a topological space, which is known as the discrete topology.
    \end{itemize}
\end{ex}

\begin{rmk}
    By De Morgan's law, closed sets are closed under arbitrary intersection and finite union, also $\emptyset, X$ are closed.

    In general, infinite union of closed sets may not be closed.
\end{rmk}

\begin{df}
    Let $M$ be a metric space and $S\subseteq M$. Define
    \[
    \Bar S := \{x\in M \mid x\text{ is a limit of }S\}
    \]
    to be the closure of $S$.
    For $x\in M$, $r > 0$, define
    \[
    B(x, r) := \{y\in M\mid d(x, y) < r\},
    \]
    which is the ball centered at $x$ with radius $r$, or the $r$-neighborhood of $x$.
\end{df}

\begin{prop}
    $\Bar S$ is closed and $B(x, r)$ is open.

    \begin{proof}
        If $S = \emptyset$, then $\Bar S = \emptyset$, which is closed.

        Suppose that $S\neq \emptyset$, and let $(x_n) \to x$ be a convergence sequence in $\Bar S$. We wish to prove that $x\in \Bar S$, i.e. there exists a sequence in $S$ that approaches $x$. Since $x_n\in \Bar S$, there exists sequence $(x_{n, k})$ in $S$ that approaches $x_n$ as $k\to \infty$. For each $n$, there exists a term $x_{n, k_n}$ satisfying $d(x_{n, k_n}, x_n) < \frac1n$, picking these terms forms a new sequence $(x_{n, k_n})$, moreover, 
        \[
        d(x_{n, k_n}, x) \leq d(x_{n, k_n}, x_n) + d(x_n, x) < \frac1n + d(x_n, x),
        \]
        which approaches $0$ as $n\to\infty$, thus $(x_{n, k_n})$ approaches $x$.

        Fix $x\in M$ and $r > 0$. Let $y \in B(x, r)$ and pick $s = r - d(x, y)$. If $z$ satisfies $d(y, z) < s$, then
        \[
        d(x, z) \leq d(x, y) + d(y, z) < d(x, y) + s = r \implies B(y, s) \subseteq B(x, r),
        \]
        so $B(x, r)$ is open.
    \end{proof}
\end{prop}

\begin{cl}
    \label{cl:smallest-closed-set}
    $\Bar S$ is the smallest closed set that conatins $S$, i.e. if $K\supseteq S$ and $K$ is closed then $K\supseteq \Bar S$.

    \begin{proof}
        $K$ contains the limit of each sequence in $K$, in particular $S\subseteq K$ so it contains all sequence in $S$ that converges in $M$, but these are precisely $\Bar S$.
    \end{proof}
\end{cl}

With the observation from Corollary \ref{cl:smallest-closed-set}, we may define the closure for topological spaces in general.

\begin{df}
    Lex $X$ be a topological space, and let $S\subseteq X$. We define $\overline S$ to be the smallest closed set that contains $S$. $\Bar S$ always exists; take
    \[
    \Bar S = \bigcap\set{E\subseteq X \mid E \text{ is closed in }X\text{ and }E\supseteq S}.
    \]
    Likewise, we also define the interior of $S$, $~int (S)$, to be the largest open set contained in $S$
\end{df}

\begin{df}
    Let $X$ be a topological space, $x\in X$. A neighborhood of $x$ is an open set containing $x$.
\end{df}

\begin{df}
    Lex $X$ be a topological space and let $(x_n)$ be a sequence in $X$. We say that $(x_n)$ converges to $x\in X$ if for all neighborhood $\mathcal U$ of $x$, $\exists N\in \N$ such that $x_n \in \mathcal U$ for all $n\geq N$.
\end{df}

\begin{ex}
    Limits in general are not unique in topological spaces. Let $X$ be a set with at least $2$ points endowed with the trivial topology. Then every sequence in $X$ converges to every point in $X$.
\end{ex}

\begin{df}
    Let $X, Y$ be topological spaces. We say a function $f: X\to Y$ continuous if for any open set $V\subseteq Y$, the preimage $f^{-1}(Y)$ is open in $X$.
\end{df}

\begin{prop}
    A function $f: X\to Y$ is continuous iff $\forall x\in M$ and any neighborhood $\mathcal V$ of $f(x)$, $\exists$ a neighborhood $\mathcal U$ of $x$ such that $f(\mathcal U) \subseteq \mathcal V$.
    \begin{proof}
        "$\implies$": Let $\mathcal V\subseteq Y$ is open. We need to show $f^{-1}(Y)$ is open. Let $x\in f^{-1}(\mathcal V)$, then $f(x) \in \mathcal V$. By definition, $\exists$ a neighborhood $\mathcal U_x$ such that $f(\mathcal U_x) \subseteq \mathcal V \leadsto \mathcal U_x \subseteq f^{-1}(\mathcal V)$. Take the union of all such $\mathcal U_x$ over $x\in f^{-1}(\mathcal V)$, then
        \begin{gather*}
            \bigcup_{x\in f^{-1}(\mathcal V)} \mathcal U_x \subseteq f^{-1}(\mathcal V) \quad\text{and}\quad \forall x \in f^{-1}(\mathcal V),\, x\in \mathcal U_x \subseteq f^{-1}(\mathcal V) \\
            \implies f^{-1}(\mathcal V) = \bigcup_{x\in f^{-1}(\mathcal V)} \mathcal U_x
        \end{gather*}
        and thus $f^{-1}(\mathcal V)$ is open.

        "$\impliedby$": Let $x\in X$ and let $\mathcal V$ be a neighborhood of $f(x)$. By definition $x\in f^{-1}(\mathcal V)$, which is an open set by assumption. Also, $f(f^{-1}(\mathcal V)) \subseteq \mathcal V$, hence $f^{-1}(\mathcal V)$ is a neighborhood of $x$ such that $f(f^{-1}(\mathcal V)) \subseteq \mathcal V$.
    \end{proof}
\end{prop}

\begin{df}
    A homeomorphism is a continuous bijection between topoloical spaces.
\end{df}

\begin{cl}
    A homeomorphism $f: X \to Y$ bijects the corresponding topologies $\mathcal T_X$ and $\mathcal T_Y$.
\end{cl}

\section{Hausdorff Space}

\begin{df}
    A topological space $X$ is said to be Hausdorff if given any pair of distinct paints $x_1, x_2 \in X$, $\exists$ neighborhoods $\mathcal U_1$ of $x_1$ and $\mathcal U_2$ of $x_2$ such that $\mathcal U_1 \cap \mathcal U_2 = \emptyset$.
\end{df}

A metric space is always a Hausdorff space.

\begin{lm}
    Let $X$ be Hausdorff.
    \begin{enumerate}[label={\alph*)}]
        \item Every one-point set is closed.
        \item If a sequence $(x_n)$ in $X$ converges, then the limit is unique.
    \end{enumerate}
    \begin{proof}
        \begin{enumerate}[label={\alph*)}]
            \item Pick $x\in X$. For any $y$ distinct from $x$, there exists disjoint neighborhoods $\mathcal U_x$ of $x$ and $\mathcal V_y$ of $y$. We have
            \[
            \set{x}^c = \bigcup_{y\in X\setminus \set{x}} \mathcal V_y
            \]
            which is an open set.

            \item 
            Suppose that $x, x'$ are distinct limits of $(x_n)$. $\exists$ disjoint neighborhoods $\mathcal U$ of $x$ and $\mathcal U'$ of $x'$. $\exists N, N'\in \N$ such that "$n\geq N'$ implies $x_n\in \mathcal U$" and "$n\geq N'$ implies $x_n \in \mathcal U'$". If $n\geq \max(N, N')$, then $x_n \in \mathcal U \cap \mathcal U' = \emptyset$, contradiction. Therefore, any converging sequence has a unique limit. 
        \end{enumerate}
    \end{proof}
\end{lm}

\section{Subspaces and Product Spaces}

\begin{df}
    Let $X$ be a topological space, let $A\subseteq X$. We define the subspace topology $\mathcal T_A$ of $A$ by
    \[
    \mathcal T_A = \set{\mathcal U\subseteq A\mid \mathcal U = A\cap \mathcal V \text{ for some open }\mathcal V\subseteq X}.
    \]
\end{df}

\begin{rmk}
    Openness and closedness are not just properties of a set itself, but rather a set in a relation to a particular topological space.
\end{rmk}

\begin{prop}
    Let $M$ be a metric space, and let $N\subseteq M$ be a nonempty subset. The subspace topology on $N$ is the same as the metric topology obtained by restricting the metric of $M$ to $N$.
    \begin{proof}
        Suppose that $\mathcal V$ is an open set in $M$ and let $\mathcal U = N\cap \mathcal V \in \mathcal T_N$. 
        
        We first to prove that $\mathcal U$ belongs to the metric toplogy of $N$. Let $x\in \mathcal U$. Since $x\in N\cap \mathcal V \subseteq \mathcal V$, there exists $r > 0$ such that $B_M(x, r)\subseteq \mathcal V$, and
        \[
        B_N(x, r) = N\cap B_M(x, r) \subseteq N\cap \mathcal V = \mathcal U,
        \]
        hence $\mathcal U$ is open in $N$ (in the metric topology).

        Conversely, let $\mathcal U$ be an open set in the metric topology of $N$. $\forall x\in \mathcal U$, $\exists r_x > 0$ such that $B_N(x, r_x) \subseteq \mathcal U$. Note that
        \[
        \mathcal U = \bigcup_{x\in\mathcal U} B_N(x, r_x) = \bigcup_{x\in\mathcal U} N \cap B_M(x, r_x) = N\cap \bigcup_{x\in\mathcal U} B_M(x, r_x),
        \]
        hence $\mathcal U$ belongs to the subspace topology of $N$.
    \end{proof}
\end{prop}

\begin{df}
    Let $X$ be a set. A basis in $X$ is a collection $\mathcal B$ of subsets of $X$ satsifying
    \begin{enumerate}[label={\alph*)}]
        \item Every element of $X$ is in some element in $\mathcal B$. That is,
        \[
        X = \bigcup_{B\in \mathcal B} B.
        \]
        \item If $B_1, B_2 \in \mathcal B$ and $x\in B_1 \cap B_2$, there exists $B_3 \in \mathcal B$ such that $x\in B_3 \subseteq B_1\cap B_2$.
    \end{enumerate}
\end{df}

\begin{df}
    Given $X$ and a collection $\mathcal B$ of subsets of $X$, we say that $\mathcal U\subseteq X$ satisfies the basis criterion with respect to $\mathcal B$ if $\forall x \in \mathcal U$, $\exists B \in \mathcal B$ such that $x\in B\subseteq \mathcal U$
\end{df}

\begin{lm}
    Suppose that $\mathcal B$ is a basis in $X$, and let $\mathcal T$ be the collection of all unions of elements of $\mathcal B$. Then $\mathcal T$ is precisely the collection of all subsets of $X$ that satisfy the basis criterion w.r.t. $\mathcal B$.
    \begin{proof}
        Let $\mathcal U\subseteq X$. Suppose that $\mathcal U$ satisfies the basis criterion. Let
        \[
        \mathcal V = \bigcup\set{B\in\mathcal B\mid B\subseteq\mathcal U},
        \]
        then $\mathcal V \in \mathcal T$. We want to show that $\mathcal U = \mathcal V$. Clearly, $\mathcal V\subseteq \mathcal U$. Let $x\in \mathcal U$. Since $\mathcal U$ satisfies the basis criterion, $\exists B\in\mathcal B$ such that $x\in B\subseteq \mathcal U$, so $x\in \mathcal V$, therefore $\mathcal U\subseteq \mathcal V$.

        Conversely, suppose that $\mathcal U\in \mathcal T$, then $\mathcal U$ is a union of elements of $\mathcal B$, say $\mathcal U = \bigcup_{B\in\mathcal A}B$ where $\mathcal A\subseteq \mathcal B$. For any $x\in\mathcal U$, $x\in B$ for some $B\in \mathcal A$, also $B\subseteq \mathcal U$, so $\mathcal U$ satisfies the basis criterion.
    \end{proof}
\end{lm}

\begin{prop}
    Let $\mathcal B$ be a basis in $X$, and let $\mathcal T$ be collection of all unions of elements of $\mathcal B$. Then $\mathcal T$ is a topology on $X$. This is called the topology generated by $\mathcal B$.
    \begin{proof}
        % TODO
    \end{proof}
\end{prop}
\appendix

\chapter{}

\section{Separation Axioms}

\begin{df}
    Let $X$ be a topological space.
    \begin{itemize}
        \item $X$ is $T_0$ if for all distinct $x, y\in X$, $\exists$ a neighborhood of one of them that does not contain the other.
        \item $X$ is $T_1$ if for all distinct $x, y\in X$, $\exists$ a neighborhood of $x$ that does not contain $y$, and $\exists$ a neighborhood of $y$ that does not contain $x$.
        \item $X$ is $T_2$ (Hausdorff) if for all distinct $x, y\in X$, $\exists$ a neighborhood of $x$ and a neighborhood of $y$ that are disjoint.
    \end{itemize}
\end{df}

\begin{thm}
    A topological space $X$ is $T_1$ iff every one-point set is closed.
\end{thm}

\begin{df}
    Let $X$ be a topological space. We say that $X$ is normal if for every pair of disjoint closed sets $A, B\subseteq X$, $\exists$ disjoint open sets $\mathcal U, \mathcal V$ such that $A\subseteq \mathcal U$ and $B\subseteq \mathcal V$.
\end{df}

\begin{df}
    Let $X$ be a topological space. We say that $X$ is completely regular if for every closed set $A\subseteq X$ and every point $x\in X\setminus A$, $\exists$ a continuous function $f:X\to[0,1]$ such that $f(x)=1$ and $f|_A=0$.
\end{df}

\begin{df}
    Let $X$ be a topological space. 
    \begin{itemize}
        \item $X$ is $T_3$ if $X$ is $T_1$ and regular.
        \item $X$ is $T_{3\frac 1 2}$ if $X$ is $T_1$ and completely regular.
        \item $X$ is $T_4$ if $X$ is $T_1$ and normal.
    \end{itemize}
\end{df}

\begin{ex}
    We'll show that
    \[
    T_4 \subsetneq T_{3\frac 1 2} \subsetneq T_3 \subsetneq T_2 \subsetneq T_1 \subsetneq T_0,
    \]
    by providing examples that separate each pair of consecutive classes.
    \begin{itemize}
        \item $T_1 \subsetneq T_0$: Let $X = \set{a, b}$, and let $\mathcal T = \set{\emptyset, X, \set{a}}$. Then $(X, \mathcal T)$ is $T_0$ but not $T_1$. This construction is called the Sierpinski space.
        \item $T_2 \subsetneq T_1$: Let $X = \R$ with the cofinite topology, i.e. $\mathcal T = \set{\emptyset} \cup \set{U\subseteq \R \mid U^c \text{ is finite}}$. Then $(X, \mathcal T)$ is $T_1$ but not $T_2$.
        \item $T_3 \subsetneq T_2$: Let $X = \R$. Consider the $K$-topology on $\R$, i.e. 
        \[
        \mathcal T = \set{\emptyset} \cup \set{U\subseteq \R \mid U = V\setminus K, V \text{ is open in the usual topology}, K\subseteq \set{\frac1n \mid n\in \N}}.
        \]
        Then $(X, \mathcal T)$ is $T_2$ but not $T_3$. This is because $\set{\frac1n \mid n\in \N}$ is closed in $\mathcal T$.
        \item $T_{3\frac12} \subsetneq T_3$: See [Muukres, Topology, Second Edition, P.214].
    \end{itemize}
    For $T_4 \subsetneq T_{3\frac 1 2}$, we first prove the following proposition:
    \begin{prop}
        For $k = 0, 1, 2, 3, 3\frac12$, $T_k$ is closed under subspace topology, while $T_4$ is not.
        \begin{proof}
            Consider the Sorgenfrey line $\R_l = (\R, \mathcal T_S)$, where $\mathcal T_S$ is generated by the basis
            \[
            \set{[a, b) \mid a < b, a, b\in \R}.
            \]
            Clearly, $\R_l$ is $T_4$. Let $\Delta = \set{(x, x) \mid x\in \R} \subseteq \R_l \times \R_l$. We claim that $\Delta$ is not normal in the subspace topology.
        \end{proof}
    \end{prop}
\end{ex}

\section{Euclid's Theorem}

\begin{thm}[Euclid's Theorem]
    There are infinitely many prime numbers.
    \begin{proof}
        Consider a topology on $\Z$. For each $a\in \Z$ and $b\in \N$, let
        \[
        N_{a, b} = \set{a + nb \mid n\in \Z}.
        \]
        We say a set $\mathcal U\subseteq \Z$ is open if for every $a\in \mathcal U$, $\exists b\in \N$ such that $N_{a, b}\subseteq \mathcal U$. 

        It is easy to verify that the union of any collection of open sets is open. The intersection of two open sets is also open, since if $a\in \mathcal U_1\cap \mathcal U_2$, then $\exists b_1, b_2\in \N$ such that $N_{a, b_1}\subseteq \mathcal U_1$ and $N_{a, b_2}\subseteq \mathcal U_2$, hence $N_{a, \mathrm{lcm}(b_1, b_2)}\subseteq \mathcal U_1\cap \mathcal U_2$. Thus this is indeed a topology.

        Note that $N_{a, b}$ is also closed, since
        \[
        N_{a, b} = \Z \setminus \bigcup_{i=1}^{b-1} N_{a+i, b}.
        \]
        Since any number $n\neq \pm 1$ has a prime divisor, we have
        \[
        \Z\setminus \set{-1, 1} = \bigcup_{p\text{ is prime}} N_{0, p}.
        \]
        If there are only finitely many primes $p_1, p_2, \ldots, p_k$, then $\Z\setminus \set{-1, 1}$ is a finite union of closed sets, hence closed. This implies that $\set{-1, 1}$ is open, which is a contradiction since any nonempty open set in this topology is infinite.
    \end{proof}
\end{thm}

\section{Quotient Topology}

\begin{df}
    Let $X, Y$ be topological spaces. A surjective map $p: X\to Y$ is called a quotient map if a set $U\subseteq Y$ is open iff $p^{-1}(U)$ is open in $X$.
\end{df}

\begin{df}
    A subset $C\subseteq X$ is called saturated w.r.t. a map $p: X\to Y$ if $C = p^{-1}(S)$ for some $S\subseteq Y$.
\end{df}

\begin{df}
    A mapping $p: X\to Y$ is called open (resp. closed) if for every open (resp. closed) set $U\subseteq X$, $p(U)$ is open (resp. closed) in $Y$.
\end{df}

\begin{prop}
    The following are equivalent:
    \begin{enumerate}[label={(\alph*)}]
        \item $p$ is a quotient map.
        \item $p$ is surjective, continuous, and maps saturated open sets to open sets.
        \item $p$ is surjective, continuous, and is either open or closed.
    \end{enumerate}
\end{prop}

\begin{df}
    Let $X$ be a topological space, $A$ be a set, and $p: X\to A$ be a surjective map. Then there is a unique topology $\mathcal T$ on $A$ such that $p: (X, \mathcal T_X)\to (A, \mathcal T)$ is a quotient map, namely
    \[
    \mathcal T = \set{U\subseteq A \mid p^{-1}(U) \text{ is open in } X}.
    \]
    This topology is called the quotient topology induced by $p$. 
\end{df}

\begin{ex}
    Let $p: \R \to \set{a, b, c}$ be defined by
    \[
    p(x) = \begin{cases}
        a, & x > 0; \\
        b, & x < 0; \\
        c, & x = 0.
    \end{cases}
    \]
    Then the quotient topology on $\set{a, b, c}$ is $\set{\emptyset, \set{a}, \set{b}, \set{a, b}, \set{a, b, c}}$.
\end{ex}

\begin{df}
    Let $X$ be a topological space, and let $X^*$ be a partition of $X$. Let $\pi: X\to X^*$ be the natural projection that maps each point to its partition. Then under the quotient topology on $X^*$ induced by $\pi$, $X^*$ is called the quotient space or the identification space.
\end{df}

\begin{ex}
    Let $X$ be the closed unit disk $\set{(x, y)\in \R^2 \mid x^2 + y^2 \leq 1}$. Let
    \[
    X^* = \set{\set{(x, y) \mid x^2 + y^2 < 1}} \cup \set{\set{(x, y) \mid x^2 + y^2 = 1}}.
    \]
    Then $X^*$ with the quotient topology is homeomorphic to the sphere $S^2$.
\end{ex}

\begin{thm}
    Let $p: X\to Y$ be a quotient map. Let $Z$ be a topological space, and $g: X\to Z$ be a map that is constant on each $p^{-1}(\set{y})$, $y\in Y$. Then, $g$ induces a map $f: Y\to Z$ such that $f\circ p = g$. Furthermore, $f$ is continuous iff $g$ is continuous, and $f$ is a quotient map iff $g$ is a quotient map.
\end{thm}

\begin{cl}
    Let $g: X\to Z$ be a surjective continuous map. Let $X^* = \set{g^{-1}(\set{z}) \mid z\in Z}$. Equip $X^*$ with the quotient topology. 
    \begin{enumerate}[label={(\alph*)}]
        \item The map $g$ induces a bijection $f: X^*\to Z$, which is a homeomorphism iff $g$ is a quotient map.
        \item If $Z$ is Hausdorff, then so is $X^*$.
    \end{enumerate}
\end{cl}

\section{Paracompactness}

\begin{df}
    Consider a collection of subset of space $X$, say $\set{U_\alpha}_{\alpha\in A}$. 
    \begin{enumerate}
        \item The collection $\set{U_\alpha}_{\alpha\in A}$ is said to be locally finite if for every $x\in X$, $\exists$ a neighborhood $V$ of $x$ such that $V$ intersects only finitely many $U_\alpha$'s.
        \item The collection $\set{U_\alpha}_{\alpha\in A}$ is said to be point finite if for every $x\in X$, $x$ belongs to only finitely many $U_\alpha$'s.
        \item Suppose $\set{U_\alpha}_{\alpha\in A}$ is an open cover of $X$. A refinement of $\set{U_\alpha}_{\alpha\in A}$ is another open cover $\set{V_\beta}_{\beta\in B}$ such that for every $\beta\in B$, $\exists \alpha\in A$ such that $V_\beta \subseteq U_\alpha$, denoted by $\set{V_\beta}_{\beta\in B} \prec \set{U_\alpha}_{\alpha\in A}$.
    \end{enumerate}
\end{df}

\begin{prop}
    Let $\set{U_\alpha}_{\alpha\in A}$ be a collection of subsets of $X$. If $\set{U_\alpha}_{\alpha\in A}$ is locally finite, then
    \begin{enumerate}[label={(\roman*)}]
        \item The collection $\set{\Bar{U_\alpha}}_{\alpha\in A}$ is also locally finite.
        \item For any $B \subseteq A$, $\bigcup_{\beta \in B} \Bar{U_\beta} = \Bar{\bigcup_{\beta\in B} U_\beta}$ is closed.
    \end{enumerate}
    \begin{pf}
        Given $x\in X$, let $V$ be a neighborhood of $x$ that intersects only finitely many $U_\alpha$'s, say $U_{\alpha_1}, U_{\alpha_2}, \ldots, U_{\alpha_n}$. Then $V$ also intersects only finitely many $\Bar{U_\alpha}$'s, namely $\Bar{U_{\alpha_1}}, \Bar{U_{\alpha_2}}, \ldots, \Bar{U_{\alpha_n}}$. Thus (i) holds.

        Let $B \subseteq A$ be given. Since $\set{U_\beta}_{\beta\in B} \subseteq \set{U_\alpha}_{\alpha\in A}$, $\set{U_\beta}_{\beta\in B}$ is also locally finite. By (i), $\set{\Bar{U_\beta}}_{\beta\in B}$ is locally finite. For any $x\in X\setminus \bigcup_{\beta\in B} \Bar{U_\beta}$, there is a neighborhood $V$ of $x$ that intersects only finitely many $\Bar{U_\beta}$'s, say $\Bar{U_{\beta_1}}, \Bar{U_{\beta_2}}, \ldots, \Bar{U_{\beta_m}}$. Note that
        \[
        U' = V \setminus \bigcup_{i=1}^m \Bar{U_{\beta_i}}
        \]
        is an open set, hence a neighborhood of $x$ that does not intersect any $\Bar{U_\beta}$. Thus $X\setminus \bigcup_{\beta\in B} \Bar{U_\beta}$ is open, i.e. $\bigcup_{\beta\in B} \Bar{U_\beta}$ is closed.
    \end{pf}
\end{prop}

\begin{lm}[Pasting Lemma]
    Let $\set{U_\alpha}_{\alpha\in A}$ be a cover of $X$ and $f: X\to Y$ be a map. If either
    \begin{enumerate}[label={(\roman*)}]
        \item each $U_\alpha$ is open, or
        \item each $U_\alpha$ is closed and $\set{U_\alpha}_{\alpha\in A}$ is locally finite,
    \end{enumerate}
    then $f$ is continuous iff $f|_{U_\alpha}$ is continuous for each $\alpha\in A$.
    \begin{pf}
        If $f$ is continuous, then obviously $f|_{U_\alpha}$ is continuous for each $\alpha\in A$.

        Conversely, suppose $f|_{U_\alpha}$ is continuous for each $\alpha\in A$. If (i) holds, then for any open set $V\subseteq Y$,
        \[
        f^{-1}(V) = \bigcup_{\alpha\in A} (f|_{U_\alpha})^{-1}(V) \cap U_\alpha
        \]
        is open, hence $f$ is continuous.

        If (ii) holds, then for any closed set $V\subseteq Y$, 
        % TODO
    \end{pf}
\end{lm}

\begin{df}
    Let $\set{U_\alpha}_{\alpha\in A}$ and $\set{V_\beta}_{\beta\in B}$ be two covers of $X$ such that $\set{V_\beta}_{\beta\in B} \prec \set{U_\alpha}_{\alpha\in A}$. We say that $\set{V_\beta}_{\beta\in B}$ is precise if $B = A$ and $V_\alpha \subseteq U_\alpha$ for each $\alpha\in A$.
\end{df}

\begin{prop}
    Suppose $\set{A_\alpha}_{\alpha\in A}$ has a locally point finite refinement $\set{B_\beta}_{\beta\in B}$. Then $\set{A_\alpha}_{\alpha\in A}$ has a precise locally finite refinement $\set{C_\alpha}_{\alpha\in A}$.
    \begin{pf}
        As a refinement, for each $\beta \in B$, the set $I_\beta = \set{\alpha\in A \mid B_\beta \subseteq A_\alpha}$ is nonempty. By the axiom of choice, there exists a choice function $\phi: B\to A$ such that $\phi(\beta) \in I_\beta$ for each $\beta\in B$. Then for each $\alpha\in A$, let $C_\alpha = \bigcup_{\beta\in \phi^{-1}(\set{\alpha})} B_\beta$. It is easy to verify that $\set{C_\alpha}_{\alpha\in A}$ is a precise refinement of $\set{A_\alpha}_{\alpha\in A}$.
    \end{pf}
\end{prop}

\begin{df}
    A topological space $X$ is said to be paracompact if every open cover of $X$ has a locally finite open refinement.
\end{df}

\begin{rmk}
    Every compact space is paracompact. The converse is not true, e.g. $\R$ is paracompact but not compact.
\end{rmk}

\begin{thm}
    A paracompact $T_2$(Hausdorff) space is $T_4$ (normal).
    \begin{pf}
        Let $X$ be a paracompact $T_2$ space. We first show that $X$ is regular. Given a closed set $F\subseteq X$ and a point $x\in X\setminus F$, for each $y\in F$, by Hausdorffness, $\exists$ a neighborhood $U_y$ of $y$ such that $x \notin \Bar{U_y}$. Then $\set{U_y}_{y\in F} \cup \set{X\setminus F}$ is an open cover of $X$. Since $X$ is paracompact, $\exists$ a locally finite open refinement $\set{V_\alpha}_{\alpha\in A}$. Let $W$ be the union of all $V_\alpha$'s that intersect $F$. Then $W$ is open, contains $F$, and does not contain $x$. Let $U = X\setminus \Bar{W}$. Then $U$ is open, contains $x$, and is disjoint from $W$. Thus $X$ is regular.
    \end{pf}
\end{thm}

\begin{df}
    Let $X, Y$ be spaces, and $f: X\to Y$ be a map. The support of $f$ is the set $~supp (f) = \Bar{\set{x\in X \mid f(x) \neq 0}}$.
\end{df}

\begin{df}
    Let $X$ be a space. A family of continuous mappings $\set{k_\alpha}_{\alpha\in A}$ from $X$ to $[0, 1]$ is called a partition of unity if
    \begin{enumerate}[label={(\roman*)}]
        \item $\set{supp(k_\alpha)}_{\alpha\in A}$ is locally finite.
        \item for each $x\in X$, $\sum_{\alpha\in A} k_\alpha(x) = 1$.
    \end{enumerate}
\end{df}

\begin{thm}
    If $X$ is paracompact and $T_2$, then every open cover $\mathcal U_{\alpha\in A}$ has a partition of unity $\set{k_\alpha}_{\alpha\in A}$ subordinate to it, i.e. $supp(k_\alpha) \subseteq U_\alpha$ for each $\alpha\in A$.
\end{thm}

\begin{prop}
    Let $X$ be a paracompact $T_2$ space, and $G, g: X \to \R$ are upper semicontinuous and lower semicontinuous functions respectively such that $G(x) < g(x)$ for each $x\in X$. Then $\exists$ a continuous function $f: X\to \R$ such that $G(x) < f(x) < g(x)$ for each $x\in X$.
\end{prop}

%%%%%%%%%%%%%%%%%%%%%%%%%%%%%%%%%%%%%%%%%%%

% \bibliographystyle{plain}
% \newdimen\origiwspc%
%   \newdimen\origiwstr%
%   \origiwspc=\fontdimen2\font% original inter word space
%   \origiwstr=\fontdimen3\font% original inter word stretch
%   \fontdimen2\font=0.25ex% inter word space
% \myfont
% \fontsize{9}{11}\selectfont
% \bibliography{references.bib}

\end{document}