\appendix

\chapter{}

\section{Separation Axioms}

\begin{df}
    Let $X$ be a topological space.
    \begin{itemize}
        \item $X$ is $T_0$ if for all distinct $x, y\in X$, $\exists$ a neighborhood of one of them that does not contain the other.
        \item $X$ is $T_1$ if for all distinct $x, y\in X$, $\exists$ a neighborhood of $x$ that does not contain $y$, and $\exists$ a neighborhood of $y$ that does not contain $x$.
        \item $X$ is $T_2$ (Hausdorff) if for all distinct $x, y\in X$, $\exists$ a neighborhood of $x$ and a neighborhood of $y$ that are disjoint.
    \end{itemize}
\end{df}

\begin{thm}
    A topological space $X$ is $T_1$ iff every one-point set is closed.
\end{thm}

\begin{df}
    Let $X$ be a topological space. We say that $X$ is normal if for every pair of disjoint closed sets $A, B\subseteq X$, $\exists$ disjoint open sets $\mathcal U, \mathcal V$ such that $A\subseteq \mathcal U$ and $B\subseteq \mathcal V$.
\end{df}

\begin{df}
    Let $X$ be a topological space. We say that $X$ is completely regular if for every closed set $A\subseteq X$ and every point $x\in X\setminus A$, $\exists$ a continuous function $f:X\to[0,1]$ such that $f(x)=1$ and $f|_A=0$.
\end{df}

\begin{df}
    Let $X$ be a topological space. 
    \begin{itemize}
        \item $X$ is $T_3$ if $X$ is $T_1$ and regular.
        \item $X$ is $T_{3\frac 1 2}$ if $X$ is $T_1$ and completely regular.
        \item $X$ is $T_4$ if $X$ is $T_1$ and normal.
    \end{itemize}
\end{df}

\begin{ex}
    We'll show that
    \[
    T_4 \subsetneq T_{3\frac 1 2} \subsetneq T_3 \subsetneq T_2 \subsetneq T_1 \subsetneq T_0,
    \]
    by providing examples that separate each pair of consecutive classes.
    \begin{itemize}
        \item $T_1 \subsetneq T_0$: Let $X = \set{a, b}$, and let $\mathcal T = \set{\emptyset, X, \set{a}}$. Then $(X, \mathcal T)$ is $T_0$ but not $T_1$. This construction is called the Sierpinski space.
        \item $T_2 \subsetneq T_1$: Let $X = \R$ with the cofinite topology, i.e. $\mathcal T = \set{\emptyset} \cup \set{U\subseteq \R \mid U^c \text{ is finite}}$. Then $(X, \mathcal T)$ is $T_1$ but not $T_2$.
        \item $T_3 \subsetneq T_2$: Let $X = \R$. Consider the $K$-topology on $\R$, i.e. 
        \[
        \mathcal T = \set{\emptyset} \cup \set{U\subseteq \R \mid U = V\setminus K, V \text{ is open in the usual topology}, K\subseteq \set{\frac1n \mid n\in \N}}.
        \]
        Then $(X, \mathcal T)$ is $T_2$ but not $T_3$. This is because $\set{\frac1n \mid n\in \N}$ is closed in $\mathcal T$.
        \item $T_{3\frac12} \subsetneq T_3$: See [Muukres, Topology, Second Edition, P.214].
    \end{itemize}
    For $T_4 \subsetneq T_{3\frac 1 2}$, we first prove the following proposition:
    \begin{prop}
        For $k = 0, 1, 2, 3, 3\frac12$, $T_k$ is closed under subspace topology, while $T_4$ is not.
        \begin{proof}
            Consider the Sorgenfrey line $\R_l = (\R, \mathcal T_S)$, where $\mathcal T_S$ is generated by the basis
            \[
            \set{[a, b) \mid a < b, a, b\in \R}.
            \]
            Clearly, $\R_l$ is $T_4$. Let $\Delta = \set{(x, x) \mid x\in \R} \subseteq \R_l \times \R_l$. We claim that $\Delta$ is not normal in the subspace topology.
        \end{proof}
    \end{prop}
\end{ex}