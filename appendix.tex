\appendix

\chapter{}

\section{Separation Axioms}

\begin{df}
    Let $X$ be a topological space.
    \begin{itemize}
        \item $X$ is $T_0$ if for all distinct $x, y\in X$, $\exists$ a neighborhood of one of them that does not contain the other.
        \item $X$ is $T_1$ if for all distinct $x, y\in X$, $\exists$ a neighborhood of $x$ that does not contain $y$, and $\exists$ a neighborhood of $y$ that does not contain $x$.
        \item $X$ is $T_2$ (Hausdorff) if for all distinct $x, y\in X$, $\exists$ a neighborhood of $x$ and a neighborhood of $y$ that are disjoint.
    \end{itemize}
\end{df}

\begin{thm}
    A topological space $X$ is $T_1$ iff every one-point set is closed.
\end{thm}

\begin{df}
    Let $X$ be a topological space. We say that $X$ is normal if for every pair of disjoint closed sets $A, B\subseteq X$, $\exists$ disjoint open sets $\mathcal U, \mathcal V$ such that $A\subseteq \mathcal U$ and $B\subseteq \mathcal V$.
\end{df}

\begin{df}
    Let $X$ be a topological space. We say that $X$ is completely regular if for every closed set $A\subseteq X$ and every point $x\in X\setminus A$, $\exists$ a continuous function $f:X\to[0,1]$ such that $f(x)=1$ and $f|_A=0$.
\end{df}

\begin{df}
    Let $X$ be a topological space. 
    \begin{itemize}
        \item $X$ is $T_3$ if $X$ is $T_1$ and regular.
        \item $X$ is $T_{3\frac 1 2}$ if $X$ is $T_1$ and completely regular.
        \item $X$ is $T_4$ if $X$ is $T_1$ and normal.
    \end{itemize}
\end{df}

\begin{ex}
    We'll show that
    \[
    T_4 \subsetneq T_{3\frac 1 2} \subsetneq T_3 \subsetneq T_2 \subsetneq T_1 \subsetneq T_0,
    \]
    by providing examples that separate each pair of consecutive classes.
    \begin{itemize}
        \item $T_1 \subsetneq T_0$: Let $X = \set{a, b}$, and let $\mathcal T = \set{\emptyset, X, \set{a}}$. Then $(X, \mathcal T)$ is $T_0$ but not $T_1$. This construction is called the Sierpinski space.
        \item $T_2 \subsetneq T_1$: Let $X = \R$ with the cofinite topology, i.e. $\mathcal T = \set{\emptyset} \cup \set{U\subseteq \R \mid U^c \text{ is finite}}$. Then $(X, \mathcal T)$ is $T_1$ but not $T_2$.
        \item $T_3 \subsetneq T_2$: Let $X = \R$. Consider the $K$-topology on $\R$, i.e. 
        \[
        \mathcal T = \set{\emptyset} \cup \set{U\subseteq \R \mid U = V\setminus K, V \text{ is open in the usual topology}, K\subseteq \set{\frac1n \mid n\in \N}}.
        \]
        Then $(X, \mathcal T)$ is $T_2$ but not $T_3$. This is because $\set{\frac1n \mid n\in \N}$ is closed in $\mathcal T$.
        \item $T_{3\frac12} \subsetneq T_3$: See [Muukres, Topology, Second Edition, P.214].
    \end{itemize}
    For $T_4 \subsetneq T_{3\frac 1 2}$, we first prove the following proposition:
    \begin{prop}
        For $k = 0, 1, 2, 3, 3\frac12$, $T_k$ is closed under subspace topology, while $T_4$ is not.
        \begin{proof}
            Consider the Sorgenfrey line $\R_l = (\R, \mathcal T_S)$, where $\mathcal T_S$ is generated by the basis
            \[
            \set{[a, b) \mid a < b, a, b\in \R}.
            \]
            Clearly, $\R_l$ is $T_4$. Let $\Delta = \set{(x, x) \mid x\in \R} \subseteq \R_l \times \R_l$. We claim that $\Delta$ is not normal in the subspace topology.
        \end{proof}
    \end{prop}
\end{ex}

\section{Euclid's Theorem}

\begin{thm}[Euclid's Theorem]
    There are infinitely many prime numbers.
    \begin{proof}
        Consider a topology on $\Z$. For each $a\in \Z$ and $b\in \N$, let
        \[
        N_{a, b} = \set{a + nb \mid n\in \Z}.
        \]
        We say a set $\mathcal U\subseteq \Z$ is open if for every $a\in \mathcal U$, $\exists b\in \N$ such that $N_{a, b}\subseteq \mathcal U$. 

        It is easy to verify that the union of any collection of open sets is open. The intersection of two open sets is also open, since if $a\in \mathcal U_1\cap \mathcal U_2$, then $\exists b_1, b_2\in \N$ such that $N_{a, b_1}\subseteq \mathcal U_1$ and $N_{a, b_2}\subseteq \mathcal U_2$, hence $N_{a, \mathrm{lcm}(b_1, b_2)}\subseteq \mathcal U_1\cap \mathcal U_2$. Thus this is indeed a topology.

        Note that $N_{a, b}$ is also closed, since
        \[
        N_{a, b} = \Z \setminus \bigcup_{i=1}^{b-1} N_{a+i, b}.
        \]
        Since any number $n\neq \pm 1$ has a prime divisor, we have
        \[
        \Z\setminus \set{-1, 1} = \bigcup_{p\text{ is prime}} N_{0, p}.
        \]
        If there are only finitely many primes $p_1, p_2, \ldots, p_k$, then $\Z\setminus \set{-1, 1}$ is a finite union of closed sets, hence closed. This implies that $\set{-1, 1}$ is open, which is a contradiction since any nonempty open set in this topology is infinite.
    \end{proof}
\end{thm}

\section{Quotient Topology}

\begin{df}
    Let $X, Y$ be topological spaces. A surjective map $p: X\to Y$ is called a quotient map if a set $U\subseteq Y$ is open iff $p^{-1}(U)$ is open in $X$.
\end{df}

\begin{df}
    A subset $C\subseteq X$ is called saturated w.r.t. a map $p: X\to Y$ if $C = p^{-1}(S)$ for some $S\subseteq Y$.
\end{df}

\begin{df}
    A mapping $p: X\to Y$ is called open (resp. closed) if for every open (resp. closed) set $U\subseteq X$, $p(U)$ is open (resp. closed) in $Y$.
\end{df}

\begin{prop}
    The following are equivalent:
    \begin{enumerate}[label={(\alph*)}]
        \item $p$ is a quotient map.
        \item $p$ is surjective, continuous, and maps saturated open sets to open sets.
        \item $p$ is surjective, continuous, and is either open or closed.
    \end{enumerate}
\end{prop}

\begin{df}
    Let $X$ be a topological space, $A$ be a set, and $p: X\to A$ be a surjective map. Then there is a unique topology $\mathcal T$ on $A$ such that $p: (X, \mathcal T_X)\to (A, \mathcal T)$ is a quotient map, namely
    \[
    \mathcal T = \set{U\subseteq A \mid p^{-1}(U) \text{ is open in } X}.
    \]
    This topology is called the quotient topology induced by $p$. 
\end{df}

\begin{ex}
    Let $p: \R \to \set{a, b, c}$ be defined by
    \[
    p(x) = \begin{cases}
        a, & x > 0; \\
        b, & x < 0; \\
        c, & x = 0.
    \end{cases}
    \]
    Then the quotient topology on $\set{a, b, c}$ is $\set{\emptyset, \set{a}, \set{b}, \set{a, b}, \set{a, b, c}}$.
\end{ex}

\begin{df}
    Let $X$ be a topological space, and let $X^*$ be a partition of $X$. Let $\pi: X\to X^*$ be the natural projection that maps each point to its partition. Then under the quotient topology on $X^*$ induced by $\pi$, $X^*$ is called the quotient space or the identification space.
\end{df}

\begin{ex}
    Let $X$ be the closed unit disk $\set{(x, y)\in \R^2 \mid x^2 + y^2 \leq 1}$. Let
    \[
    X^* = \set{\set{(x, y) \mid x^2 + y^2 < 1}} \cup \set{\set{(x, y) \mid x^2 + y^2 = 1}}.
    \]
    Then $X^*$ with the quotient topology is homeomorphic to the sphere $S^2$.
\end{ex}

\begin{thm}
    Let $p: X\to Y$ be a quotient map. Let $Z$ be a topological space, and $g: X\to Z$ be a map that is constant on each $p^{-1}(\set{y})$, $y\in Y$. Then, $g$ induces a map $f: Y\to Z$ such that $f\circ p = g$. Furthermore, $f$ is continuous iff $g$ is continuous, and $f$ is a quotient map iff $g$ is a quotient map.
\end{thm}

\begin{cl}
    Let $g: X\to Z$ be a surjective continuous map. Let $X^* = \set{g^{-1}(\set{z}) \mid z\in Z}$. Equip $X^*$ with the quotient topology. 
    \begin{enumerate}[label={(\alph*)}]
        \item The map $g$ induces a bijection $f: X^*\to Z$, which is a homeomorphism iff $g$ is a quotient map.
        \item If $Z$ is Hausdorff, then so is $X^*$.
    \end{enumerate}
\end{cl}

\section{Paracompactness}

\begin{df}
    Consider a collection of subset of space $X$, say $\set{U_\alpha}_{\alpha\in A}$. 
    \begin{enumerate}
        \item The collection $\set{U_\alpha}_{\alpha\in A}$ is said to be locally finite if for every $x\in X$, $\exists$ a neighborhood $V$ of $x$ such that $V$ intersects only finitely many $U_\alpha$'s.
        \item The collection $\set{U_\alpha}_{\alpha\in A}$ is said to be point finite if for every $x\in X$, $x$ belongs to only finitely many $U_\alpha$'s.
        \item Suppose $\set{U_\alpha}_{\alpha\in A}$ is an open cover of $X$. A refinement of $\set{U_\alpha}_{\alpha\in A}$ is another open cover $\set{V_\beta}_{\beta\in B}$ such that for every $\beta\in B$, $\exists \alpha\in A$ such that $V_\beta \subseteq U_\alpha$, denoted by $\set{V_\beta}_{\beta\in B} \prec \set{U_\alpha}_{\alpha\in A}$.
    \end{enumerate}
\end{df}

\begin{prop}
    Let $\set{U_\alpha}_{\alpha\in A}$ be a collection of subsets of $X$. If $\set{U_\alpha}_{\alpha\in A}$ is locally finite, then
    \begin{enumerate}[label={(\roman*)}]
        \item The collection $\set{\Bar{U_\alpha}}_{\alpha\in A}$ is also locally finite.
        \item For any $B \subseteq A$, $\bigcup_{\beta \in B} \Bar{U_\beta} = \Bar{\bigcup_{\beta\in B} U_\beta}$ is closed.
    \end{enumerate}
    \begin{pf}
        Given $x\in X$, let $V$ be a neighborhood of $x$ that intersects only finitely many $U_\alpha$'s, say $U_{\alpha_1}, U_{\alpha_2}, \ldots, U_{\alpha_n}$. Then $V$ also intersects only finitely many $\Bar{U_\alpha}$'s, namely $\Bar{U_{\alpha_1}}, \Bar{U_{\alpha_2}}, \ldots, \Bar{U_{\alpha_n}}$. Thus (i) holds.

        Let $B \subseteq A$ be given. Since $\set{U_\beta}_{\beta\in B} \subseteq \set{U_\alpha}_{\alpha\in A}$, $\set{U_\beta}_{\beta\in B}$ is also locally finite. By (i), $\set{\Bar{U_\beta}}_{\beta\in B}$ is locally finite. For any $x\in X\setminus \bigcup_{\beta\in B} \Bar{U_\beta}$, there is a neighborhood $V$ of $x$ that intersects only finitely many $\Bar{U_\beta}$'s, say $\Bar{U_{\beta_1}}, \Bar{U_{\beta_2}}, \ldots, \Bar{U_{\beta_m}}$. Note that
        \[
        U' = V \setminus \bigcup_{i=1}^m \Bar{U_{\beta_i}}
        \]
        is an open set, hence a neighborhood of $x$ that does not intersect any $\Bar{U_\beta}$. Thus $X\setminus \bigcup_{\beta\in B} \Bar{U_\beta}$ is open, i.e. $\bigcup_{\beta\in B} \Bar{U_\beta}$ is closed.
    \end{pf}
\end{prop}

\begin{lm}[Pasting Lemma]
    Let $\set{U_\alpha}_{\alpha\in A}$ be a cover of $X$ and $f: X\to Y$ be a map. If either
    \begin{enumerate}[label={(\roman*)}]
        \item each $U_\alpha$ is open, or
        \item each $U_\alpha$ is closed and $\set{U_\alpha}_{\alpha\in A}$ is locally finite,
    \end{enumerate}
    then $f$ is continuous iff $f|_{U_\alpha}$ is continuous for each $\alpha\in A$.
    \begin{pf}
        If $f$ is continuous, then obviously $f|_{U_\alpha}$ is continuous for each $\alpha\in A$.

        Conversely, suppose $f|_{U_\alpha}$ is continuous for each $\alpha\in A$. If (i) holds, then for any open set $V\subseteq Y$,
        \[
        f^{-1}(V) = \bigcup_{\alpha\in A} (f|_{U_\alpha})^{-1}(V) \cap U_\alpha
        \]
        is open, hence $f$ is continuous.

        If (ii) holds, then for any closed set $V\subseteq Y$, 
        % TODO
    \end{pf}
\end{lm}

\begin{df}
    Let $\set{U_\alpha}_{\alpha\in A}$ and $\set{V_\beta}_{\beta\in B}$ be two covers of $X$ such that $\set{V_\beta}_{\beta\in B} \prec \set{U_\alpha}_{\alpha\in A}$. We say that $\set{V_\beta}_{\beta\in B}$ is precise if $B = A$ and $V_\alpha \subseteq U_\alpha$ for each $\alpha\in A$.
\end{df}

\begin{prop}
    Suppose $\set{A_\alpha}_{\alpha\in A}$ has a locally point finite refinement $\set{B_\beta}_{\beta\in B}$. Then $\set{A_\alpha}_{\alpha\in A}$ has a precise locally finite refinement $\set{C_\alpha}_{\alpha\in A}$.
    \begin{pf}
        As a refinement, for each $\beta \in B$, the set $I_\beta = \set{\alpha\in A \mid B_\beta \subseteq A_\alpha}$ is nonempty. By the axiom of choice, there exists a choice function $\phi: B\to A$ such that $\phi(\beta) \in I_\beta$ for each $\beta\in B$. Then for each $\alpha\in A$, let $C_\alpha = \bigcup_{\beta\in \phi^{-1}(\set{\alpha})} B_\beta$. It is easy to verify that $\set{C_\alpha}_{\alpha\in A}$ is a precise refinement of $\set{A_\alpha}_{\alpha\in A}$.
    \end{pf}
\end{prop}

\begin{df}
    A topological space $X$ is said to be paracompact if every open cover of $X$ has a locally finite open refinement.
\end{df}

\begin{rmk}
    Every compact space is paracompact. The converse is not true, e.g. $\R$ is paracompact but not compact.
\end{rmk}

\begin{thm}
    A paracompact $T_2$(Hausdorff) space is $T_4$ (normal).
    \begin{pf}
        Let $X$ be a paracompact $T_2$ space. We first show that $X$ is regular. Given a closed set $F\subseteq X$ and a point $x\in X\setminus F$, for each $y\in F$, by Hausdorffness, $\exists$ a neighborhood $U_y$ of $y$ such that $x \notin \Bar{U_y}$. Then $\set{U_y}_{y\in F} \cup \set{X\setminus F}$ is an open cover of $X$. Since $X$ is paracompact, $\exists$ a locally finite open refinement $\set{V_\alpha}_{\alpha\in A}$. Let $W$ be the union of all $V_\alpha$'s that intersect $F$. Then $W$ is open, contains $F$, and does not contain $x$. Let $U = X\setminus \Bar{W}$. Then $U$ is open, contains $x$, and is disjoint from $W$. Thus $X$ is regular.
    \end{pf}
\end{thm}

\begin{df}
    Let $X, Y$ be spaces, and $f: X\to Y$ be a map. The support of $f$ is the set $~supp (f) = \Bar{\set{x\in X \mid f(x) \neq 0}}$.
\end{df}

\begin{df}
    Let $X$ be a space. A family of continuous mappings $\set{k_\alpha}_{\alpha\in A}$ from $X$ to $[0, 1]$ is called a partition of unity if
    \begin{enumerate}[label={(\roman*)}]
        \item $\set{supp(k_\alpha)}_{\alpha\in A}$ is locally finite.
        \item for each $x\in X$, $\sum_{\alpha\in A} k_\alpha(x) = 1$.
    \end{enumerate}
\end{df}

\begin{thm}
    If $X$ is paracompact and $T_2$, then every open cover $\mathcal U_{\alpha\in A}$ has a partition of unity $\set{k_\alpha}_{\alpha\in A}$ subordinate to it, i.e. $supp(k_\alpha) \subseteq U_\alpha$ for each $\alpha\in A$.
\end{thm}

\begin{prop}
    Let $X$ be a paracompact $T_2$ space, and $G, g: X \to \R$ are upper semicontinuous and lower semicontinuous functions respectively such that $G(x) < g(x)$ for each $x\in X$. Then $\exists$ a continuous function $f: X\to \R$ such that $G(x) < f(x) < g(x)$ for each $x\in X$.
\end{prop}

\section{More on Cantor Set}

\begin{lm}
    \label{lm:more-on-cantor-1}
    If $M$ is a nonempty compact metric space, then for every $\epsilon > 0$, $M$ can be expressed as a finite union of nonempty compact sets of diameter less than $\epsilon$.
    \begin{pf}
        For each $x\in M$, let $U_x = B(x, \epsilon/2)$. Then $\set{U_x}_{x\in M}$ is an open cover of $M$. Take its finite subcovering, then take the closure of each set.
    \end{pf}
\end{lm}

\begin{thm}[Cantor Surjection Theorem]
    Given any compact metric space $M$, there is a continuous surjection $f: C \to M$, where $C$ is the Cantor set.
    \begin{pf}
        Let $W(n)$ be the set of binary strings with $0$ and $2$ of length $n$. By Lemma \ref{lm:more-on-cantor-1}, we can divide $M$ into a finite number of nonempty compact sets of diameter at most $1$. Say there are $m \leq 2^{n_1}$ such sets, indexed by $W(n_1)$. Denote these sets by $S_\alpha$, $\alpha \in W(n_1)$. Then, divide $S_\alpha$ into finitely many pieces of diameter at most $1/2$, say there are at most $2^{n_2}$ pieces, for each $\alpha \in W(n_1)$. Label the pieces of $S_{\alpha_1}$ by $S_{\alpha_1\alpha_2}$, $\alpha_2 \in W(n_2)$. Continuing this process, we obtain a sequence of integers $n_1, n_2, \ldots$, and a family of nonempty compact sets $S_{\alpha_1\alpha_2\ldots\alpha_k}$ of diameter at most $2^{-k+1}$. For each infinite binary string $w = \alpha_1\alpha_2\ldots$ where $\alpha_i \in W(n_i)$, define
        \[
        S_w = \bigcap_{k=1}^\infty S_{\alpha_1\alpha_2\ldots\alpha_k}.
        \]
        Since $S_{\alpha_1\alpha_2\ldots\alpha_k}$ is a decreasing sequence of nonempty compact sets and the diameter approaches $0$, by \ref{prop:nested-closed-sets}, $S_w$ is a singleton. Define $f: C \to M$ by $f(c_w) = $ the unique element in $S_w$, and $c_w$ is the element in $C$ denoted by the string $w$, i.e.
        \[
        c_w = \sum_{i=1}^\infty \frac{w_i}{3^i}.
        \]
        It is easy to verify that $f$ is sequentially continuous, hence proved.
    \end{pf}
\end{thm}

\begin{df}
    We say $M$ is a Cantor space if $M$ is nonempty, compact, perfect, and totally disconnected.
\end{df}

\begin{thm}[Moore-Kline Theorem]
    Every Cantor space is homeomorphic to the Cantor set.
    \begin{pf}
        Let $M$ be a Cantor space. By the Cantor Surjection Theorem, $\exists$ a continuous surjection $f: C \to M$. We can divide a closed set $S \subseteq M$ into finitely many disjoint clopen sets: for $x \in S$, by totally disconnectedness, $\exists$ a clopen set $U$ such that $x \in U_x \subseteq B_{\epsilon/2}(x)$. Then $\set{U_x}_{x \in S}$ is an open cover of $S$. Take a finite subcovering $U_{x_1}, U_{x_2}, \ldots, U_{x_n}$, then
        \[
        \set{U_{x_1}, U_{x_2} \setminus U_{x_1}, U_{x_3} \setminus (U_{x_1} \cup U_{x_2}), \ldots, U_{x_n} \setminus \bigcup_{i=1}^{n-1} U_{x_i}}
        \]
        forms a disjoint clopen cover of $S$. Also, we can increase the number of nonempty compact sets by just one: let $U$ be a clopen subset of $M$. Since $M$ is perfect, so is $U$, thus containing at least two points. Pick two distinct points $x, y \in U$, then $\exists$ a clopen set $V$ such that $x \in V \subseteq U$ and $y \notin V$. Then $\set{V, U \setminus V}$ is a disjoint clopen cover of $U$. By repeating this process, we can divide $M$ into $2^k$ disjoint clopen sets for any $k \in \N$. Hence, we may remove the non-injective part of $f$ step by step, and finally obtain a continuous bijection from a closed subset of $C$ to $M$. Since $C$ is compact and $M$ is Hausdorff, this bijection is a homeomorphism.
    \end{pf}
\end{thm}

\section{Local Compactness and Topological Groups}

Let $f: \R \to \R$ be a smooth function such that for all $x \in \R$, there exists $n \in \N$ such that $f^{(n)}(x) = 0$. We want to show that $f$ is a polynomial.

\begin{df}
    Let $X$ be a topological space. $X$ is locally compact if for every $x \in X$, $\exists$ a neighborhood $U$ of $x$ such that $\Bar{U}$ is compact.
\end{df}

\begin{thm}
    Let $X$ be Hausdorff. The followings are equivalent:
    \begin{enumerate}
        \item $X$ is locally compact.
        \item For every $x \in X$, $\exists$ a neighborhood basis $\mathcal B$ at $x$ such that for every $U \in \mathcal B$, $\Bar{U}$ is compact.
        \item For all compact subsets $K \subseteq X$ and open set $U$ containing $K$, $\exists$ an open set $V$ such that $K \subseteq V \subseteq \Bar{V} \subseteq U$ and $\Bar{V}$ is compact.
        \item For every $x \in X$ and every neighborhood $U$ of $x$, $\exists$ a neighborhood $V$ of $x$ such that $\Bar{V}$ is compact and $\Bar{V} \subseteq U$.
    \end{enumerate}
\end{thm}

\begin{thm}[Baire Category Theorem]
    Let $(M, d)$ be a complete metric space. Then the countable intersection of dense open sets is still dense.
    \begin{proof}
        Let $(D_n)_{n\in \N}$ be a sequence of dense open sets in $M$. Let $B_1$ be a open nonempty set in $M$. Since $D_1$ is dense, $B_1 \cap D_1$ is nonempty. Let $x_1 \in B_1 \cap D_1$. Then $\exists r_1 \in (0, 1)$ such that $\Bar{B(x_1, r_1)} \subseteq B_1 \cap D_1$. Since $D_2$ is dense, $B(x_1, r_1) \cap D_2$ is nonempty. Let $x_2 \in B(x_1, r_1) \cap D_2$. Then $\exists r_2 \in (0, 1/2)$ such that $\Bar{B(x_2, r_2)} \subseteq B(x_1, r_1) \cap D_2$. Continuing this process, we obtain a sequence of points $(x_n)_{n\in \N}$ and a sequence of positive numbers $(r_n)_{n\in \N}$ such that for each $n\in \N$,
        \[
        x_n \in D_n, \quad r_n < \frac{1}{n}, \quad \Bar{B(x_n, r_n)} \subseteq B(x_{n-1}, r_{n-1}) \cap D_n.
        \]
        Note that $(x_n)_{n\in \N}$ is a Cauchy sequence, hence converges to some $x \in M$. Since $\Bar{B(x_n, r_n)}$ is closed and contains $x_m$ for all $m \geq n$, it also contains $x$. Thus $x \in D_n$ for all $n\in \N$. Also, $x \in B_1$ since $\Bar{B(x_1, r_1)} \subseteq B_1$. Therefore, $x \in B_1 \cap \bigcap_{n=1}^\infty D_n$, which shows that $\bigcap_{n=1}^\infty D_n$ is dense.
    \end{proof}
\end{thm}

\begin{df}
    A topological space $X$ is called a Baire space if the countable intersection of dense open sets is still dense.
\end{df}

\begin{thm}
    Every locally compact Hausdorff space is a Baire space.
    \begin{proof}
        Let $(D_n)_{n\in \N}$ be a sequence of dense open sets in $X$. Let $B_1$ be a nonempty open set in $X$. Since $D_1$ is dense, $B_1 \cap D_1$ is nonempty. Pick $x_1 \in B_1 \cap D_1$. Since $X$ is locally compact, $\exists$ a neighborhood $U_1$ of $x_1$ such that $\Bar{U_1}$ is compact and $\Bar{U_1} \subseteq B_1 \cap D_1$. Since $D_2$ is dense, $U_1 \cap D_2$ is nonempty. Pick $x_2 \in U_1 \cap D_2$. Again, $\exists$ a neighborhood $U_2$ of $x_2$ such that $\Bar{U_2}$ is compact and $\Bar{U_2} \subseteq U_1 \cap D_2$. Continuing this process, we obtain a sequence of points $(x_n)_{n\in \N}$ and a sequence of neighborhoods $(U_n)_{n\in \N}$ such that for each $n\in \N$,
        \[
        x_n \in D_n, \quad \Bar{U_n} \text{ is compact}, \quad \Bar{U_n} \subseteq U_{n-1} \cap D_n.
        \]
        Note that $\set{\Bar{U_n}}_{n\in \N}$ is a decreasing sequence of nonempty compact sets, hence by \ref{prop:nested-closed-sets}, $\bigcap_{n=1}^\infty \Bar{U_n} \neq \emptyset$. Let $x$ be an element in this intersection. Then $x \in D_n$ for all $n\in \N$. Also, $x \in B_1$ since $\Bar{U_1} \subseteq B_1$. Therefore, $x \in B_1 \cap \bigcap_{n=1}^\infty D_n$, which shows that $\bigcap_{n=1}^\infty D_n$ is dense.
    \end{proof}
\end{thm}

\begin{df}
    A space $X$ is called $\sigma$-compact if $X$ is a countable union of compact subsets.
\end{df}

\begin{df}
    A group $(G, \cdot)$ is called a topological group if $G$ is a topological space and the group operations (multiplication and inversion) are continuous.
\end{df}

\begin{prop}
    Let $G$ be a topological group and $a \in G$, then the maps $x \mapsto a \cdot x$, $x \mapsto x \cdot a$, and $x \mapsto x^{-1}$ are homeomorphisms.
\end{prop}

\begin{df}
    Let $X$ be a topological space and $x \in X$. A neighborhood basis at $x$ is a collection $\mathcal B$ of neighborhoods of $x$ such that for every neighborhood $U$ of $x$, $\exists V \in \mathcal B$ such that $V \subseteq U$.
\end{df}

\begin{prop}
    Let $G$ be a topological group and $N$ be a neighborhood basis at the identity $e$. Then the collections
    \[
    \set{Ux \mid x \in G, U \in N} \quad \text{and} \quad \set{xU \mid x \in G, U \in N}
    \]
    are bases for the topology of $G$.
    \begin{prop}
        Let $W$ be an open set in $G$ and $x \in W$. Then since $e \in x^{-1}W$, there exists $U \in N$ such that $U \subseteq x^{-1}W$. Thus $xU \subseteq W$, which shows that $\set{xU \mid x \in G, U \in N}$ is a basis. Similarly, $\set{Ux \mid x \in G, U \in N}$ is also a basis.
    \end{prop}
\end{prop}

\begin{prop}
    Let $G$ be a topological group, then the collection of open sets forms an ideal, i.e. for any open set $U$ and any subset $A \subseteq G$, $AU$ and $UA$ are open.
\end{prop}

\begin{thm}
    Let $G, H$ be locally compact Hausdorff topological groups, and let $f: G \to H$ be a continuous epimorphism, then $f$ is an open map whenever $G$ is $\sigma$-compact.
\end{thm}